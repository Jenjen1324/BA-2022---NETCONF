\RequirePackage{times}

\usepackage[english]{babel}
%-% \usepackage[german]{babel}
%\usepackage[ngerman]{babel}

\usepackage[utf8]{inputenc}
\usepackage[T1]{fontenc}
\usepackage{textcomp}
\usepackage{type1cm}
\usepackage[table,xcdraw]{xcolor}
\usepackage{a4}

\usepackage{graphicx}
\graphicspath{{Figures/},{Pictures/}}
\usepackage{subfigure}

\usepackage{fancyhdr}
\usepackage{fancybox}
\usepackage{hyperref}

\usepackage{float}
\usepackage{longtable}
\usepackage{paralist}
\usepackage{url}
\usepackage{lscape}
\usepackage{moreverb}

\usepackage{makecell}
\usepackage{multirow}

\usepackage{tabularx}
\usepackage{xltabular}

\usepackage{nomencl}
  \let\abbrev\nomenclature
  \renewcommand{\nomname}{List of Abbrevations}
  \setlength{\nomlabelwidth}{.25\hsize}
  \renewcommand{\nomlabel}[1]{#1 \dotfill}
  \setlength{\nomitemsep}{-\parsep}
  %For old nomencl package, uncomment this:
  \makeglossary 
  %For new nomencl package, uncomment this:
  %\makenomenclature

\usepackage[normalem]{ulem}
  \newcommand{\markup}[1]{\uline{#1}}
  
   
\usepackage{ae,aecompl}

\usepackage{enumitem}

\usepackage[first,bottomafter,light]{draftcopy}
\draftcopyName{Draft v0.1}{120}

\addtolength{\textwidth}{2cm}
\addtolength{\textheight}{2cm}
\addtolength{\oddsidemargin}{-1.0cm}
\addtolength{\evensidemargin}{-1.0cm}
\addtolength{\topmargin}{-1.5cm}

%% No Serifs: Put comment markers in front of the next three lines otherwise
\renewcommand{\ttdefault}{cmtt}
\renewcommand{\rmdefault}{phv}  % Helvetica for roman type as well as sf
\renewcommand{\ttdefault}{pcr}  % use Courier for fixed pitch, if needed

\newcommand{\?}{\discretionary{/}{}{/}}
\newcommand{\liter}[0]{/home/ruf/Lib/Bibl/}
\newcommand{\fref}[1]{\mbox{Figure~\ref{#1}}}

\pagestyle{fancy}
%%-lpr Note: 'chapters' are defined for 'book's only
%%-lpr       in articles, we make use of sections only
\renewcommand{\chaptermark}[1]{\markboth{#1}{}}
\renewcommand{\sectionmark}[1]{\markright{\thesection\ #1}}
\fancyhf{}

%\fancyhead[LE,RO]{\bfseries\thepage}
%\fancyhead[LO]{\bfseries\rightmark}
%\fancyhead[RE]{\bfseries\leftmark}
\fancyhead[R]{\leftmark}
\fancyhead[L]{\rightmark}
\fancyfoot[R]{\thepage}
\fancyfoot[C]{ZHAW SoE}
\fancyfoot[L]{Bachelor Thesis (IT)}

\renewcommand{\headrulewidth}{0.5pt}
\renewcommand{\footrulewidth}{0.5pt}
\addtolength{\headheight}{0.5pt}
\fancypagestyle{plain}{%
   \fancyhf{}
   \fancyfoot[C]{\bfseries \thepage}
   \fancyhead{}%get rid of headers on plain pages
   \renewcommand{\headrulewidth}{0pt} % an the line
}
\newcommand{\clearemptydoublepage}{\newpage{\pagestyle{empty}\cleardoublepage}}

\setlength{\parindent}{0in}
\setlength{\parskip}{3pt}

\linespread{1.15}

\hyphenation{Strea-men}
\hyphenation{Apps}
\hyphenation{Threat}
\hyphenation{Init}

\newcommand{\Appendix}[2][?]
{
  \refstepcounter{section}
  \addcontentsline{toc}{appendix}
  {
    \protect\numberline{\appendixname~\thesection} %1
  }
  {
    \flushright\large\bfseries\appendixname\ \thesection\par
    \nohypens\centering#1\par
  }
  \vspace{\baselineskip}
}

\let\margin\marginpar
\newcommand\myMargin[1]{\margin{\raggedright\scriptsize #1}}
\renewcommand{\marginpar}[1]{\myMargin{#1}}

\newcommand\CHECK{\myMargin{CHECK}}
\newcommand\NEW{\myMargin{NEW}}

\usepackage{alltt}

\newcommand{\ci}[1]{%
\fcolorbox{red}{yellow}{#1}
}%

\newcommand{\cn}[1]{%
  \fcolorbox{red}{yellow}{%
    \parbox{\textwidth}{%
      #1
    }%
  }%
}%

\usepackage[acronym,numberedsection=autolabel,section=section,automake]{glossaries}

\makeglossaries

%\useshorthands*{"}
%\addto\extrasenglish{\languageshorthands{ngerman}}

\usepackage{csquotes}
\usepackage[
backend=biber,
bibstyle=ieee,
style=numeric,
sorting=none
]{biblatex}
\addbibresource{references.bib}


\usepackage{pdfpages}
\usepackage{rotating}
\usepackage{pdflscape}

%% Adds another nesting level of numbering for sections
\setcounter{secnumdepth}{3}

\usepackage{helvet}
%\renewcommand*{\familydefault}{\sfdefault}

\usepackage{color,soul}

\usepackage{scrhack}

\usepackage[section,numbib,nottoc]{tocbibind}
\renewcommand{\listoffigures}{
    \tocsection
    \tocfile{\listfigurename}{lof}
}
\renewcommand{\listoftables}{
    \tocsection
    \tocfile{\listtablename}{lot}
}

\newcommand{\icode}[1]{``#1''}

\usepackage{listings}
\usepackage{xcolor}

\definecolor{gray}{rgb}{0.4,0.4,0.4}
\definecolor{darkblue}{rgb}{0.0,0.0,0.6}
\definecolor{cyan}{rgb}{0.0,0.6,0.6}

\definecolor{codegreen}{rgb}{0,0.6,0}
\definecolor{codegray}{rgb}{0.5,0.5,0.5}
\definecolor{codepurple}{rgb}{0.58,0,0.82}
\definecolor{backcolour}{rgb}{0.95,0.95,0.92}
\lstdefinestyle{code}{
  basicstyle=\ttfamily\footnotesize,
  columns=fullflexible,
  showstringspaces=false,
  commentstyle=\color{gray}\upshape,
    backgroundcolor=\color{backcolour},   
    keywordstyle=\color{magenta},
    numberstyle=\tiny\color{codegray},
    stringstyle=\color{codepurple},
    breakatwhitespace=false,         
    breaklines=true,                 
    captionpos=b,                    
    keepspaces=true,                 
    numbers=left,                    
    numbersep=5pt,                  
    showspaces=false,        
    showtabs=false,                  
    tabsize=2
}

\lstdefinelanguage{XML}
{
  morestring=[b]",
  morestring=[s]{>}{<},
  morecomment=[s]{<?}{?>},
  identifierstyle=\color{darkblue},
  keywordstyle=\color{cyan},
  morekeywords={xmlns,version,type}% list your attributes here
}

\lstset{
  style=code,
}