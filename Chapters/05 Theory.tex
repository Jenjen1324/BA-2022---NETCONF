\chapter{\label{theory}Theory}
\thispagestyle{fancy}


\section{OSS/BSS}

\cn{
Differentiate OSS and BSS. Where NetBox will be the core OSS and everything
connecting to it is the BSS.
}

\cn{includes inventory management}

\section{Secure credential storage}

\cn{Need for credentials in configuration}

\cn{Refer to security theory, least privileges, etc.}


\section{Network Management Systems}

% Mostly propieritary solutions
% - Cisco
% - Juniper
% - etc.

\section{Configuration Methods}

\subsection{CLI Configuration}

Probably the most common method of configuring devices is through its \acrlong{CLI}. 
Most modern networking hardware supports access to the \acrshort{CLI} through SSH,
but in cases where networking may not be available, the same interface is often also
presented in the form of a serial connection, through a dedicated port on the device.

As there is no standardization for the \acrshort{CLI} across vendors, it makes it a
difficult target for automation in a vendor agnostic space. On the other hand it
is also usually the most complete interface in terms of features for the device
you are configuring, as most vendors recognise it as the primary interface for configuration\cite{noauthor_configuration_nodate}.

There are projects such as NAPALM (Network Automation and Programmability Abstraction Layer with Multivendor support)\cite{noauthor_napalm_nodate}
which aim to solve the problem of providing a similar interface to multiple vendors,
but the functionality of such projects is severely limited in terms of interfaces provided
for configuration across vendors.

Since the \acrshort{CLI} is a text based interface, it makes it hard to validate the configuration
before deployment. Especially since the order of configuration statements often matters
(e.g. you cannot assign a VLAN before creating it), it makes it hard to stitch together
multiple templates without having an engine understand the whole configuration before sending
it off to the device.

\subsection{SNMP}

The Simple Network Management Protocol (SNMP)\cite{fedor_simple_1990} is a protocol which was originally
designed in the 1980s by a group of collaborators which sought to create a protocol which would
allow large scale-deployment of the internet.
As the name implies, it is quite a simple protocol based on a few key operations:
GetRequest, SetRequest, GetNextRequest, GetBulkRequest and some more.

In practice it works with vendor supplied ,,management information base'' (MIB) files, which identify a collection of
of object identifiers (OIDs) with a name, description and data-type. 
Along the vendor specific MIBs, there are also some standard collections,
which try to define generic information which can be queried.
So for example the OID ,,1.3.6.1.2.1.1.1'' corresponds to ,,iso.org.dod.internet.mgmt.mib.system.sysDescr''
which represents a textual description of the device which is queried.

SNMP is still largely used today for monitoring, as it is a (as the name implies) simple and light protocol
which usually doesn't generate a lot of overhead. There are some shortcomings though when it comes to
structured configuration data, especially with write/set operations. Altough data types can now be verified
using the MIB files, a configuration section as a whole still cannot be verified before sending it to a device,
as dependencies between configuration nodes cannot be specified in the MIB format. Also handling data
in lists is complex, as when one wants to edit a specific node, the whole list first has to be searched
with get requests in order to identify the correct offset which is to be modified.

\subsection{NETCONF}

\cn{Briefly explain the start of the NETCONF protocol}
\cn{Transition into NETCONF/YANG}

\subsection{NETCONF/YANG}

\cn{Explain yang modeling and why it helps}
\cn{refer to useful projects: sysrepo, netopeer2, yangexplorer, yanglib, pyang, ncclient}

\cn{mention RESTCONF}

\subsection{HTTP APIs}

\cn{Completely vendor specific}
\cn{Often not available at all}

\section{User Interface for Engineers}

\cn{Existing user interfaces}

\cn{Gathering information from documentation}

\cn{Gathering information from Yang Models}

\section{Case Study - Init7 (Schweiz) AG}

\cn{Current state: Not a lot of automation}

\cn{If configuration from OSS - forces correct documentation -> documentation implies configuration}


