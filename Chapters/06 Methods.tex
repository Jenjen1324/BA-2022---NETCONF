\chapter{\label{methods}Methods}
\thispagestyle{fancy}


\section{Application structure}

To bridge the gap between the inventory management provided by NetBox and communicating with the actual hardware,
the solution is divided into layers.
In order to facilitate authentication and authorization there is the ,,Credential Access Manager'' which communicates
with an external credential store. The access manager provides a user interface to enter and manage credentials, as well
as restricting who has access to these credentials by integrating into the NetBox permission system.
The ,,Device Context Provider'' enables a user to define what information for a specific device shall be pulled.
Using the existing GraphQL API provided by NetBox, the user can easily identify what information is available and how it is
accessed. After the context data is defined it will subsequently be used in the ,,Configuration Template Editor''
where the user writes a template using the previously defined context data. This template can then be applied to
a class of devices utilizing constraints such as tags or locations.
Lastly, the ,,Configuration Transport'' module pulls the configuration template and the credentials for a specific device
and pushes it to the actual device.

