\chapter{\label{methods}Methods}
\thispagestyle{fancy}


To gain knowledge and an understanding of the subject, literature was researched and
the Init7 network engineering team was questioned. As only a
prototype was created to the feasibility, a waterfall-like structure was used for
the planning of the implementation.

\section{Literature Research Method}

While documentation regarding standardized network protocols is fairly easy to find,
open research specific to configuration management is rather limited, because solutions
mostly exist in an enterprise context.

Research literature was still found through search platforms such as Google Scholar and
ZHAW Swisscovery. Focused was on specific search terms:

\begin{itemize}
  \item NETCONF
  \item YANG
  \item Network Management System
  \item Network Automation
  \item Network Configuration Protocol
\end{itemize}

The found articles were filtered for relevance and the rest was thoroughly studied.

Although some useful resources were identified, most information was found through
blog articles, vendor documentation, open source projects and testing methods on real
devices.

\subsection{Init7 Network Engineering Team}

As the industry partner for this thesis, Init7 provided assistance and insight into
networking infrastructure through their network engineers.
The engineering team provided information about the general infrastructure within Init7, 
how the current business processes are implemented, in-depth information about
networking, as well as general support when questions arose.

\section{\label{method-soft}Software Engineering Method}

The decision to make a plugin for NetBox was made during the definition
of the thesis. This decision was based on that NetBox was already used
within Init7, is open source and provides a stable plugin framework.
As the implementation is a prototype to prove feasibility of the
configuration method, no iterative development was needed.
In order to have a structured timeline and implementation plan,
a waterfall like software planning method was chosen.
This means, the development process was split into the following phases:

\begin{enumerate}
  \item Requirements analysis including a ``definition of done''. This is further detailed
        in the evaluation method.
  \item Abstract design of the architecture of the plugin
  \item Implement the planned design
  \item Test the implementation with the planned use cases from the requirements stage
\end{enumerate}

\section{Evaluation Method}

In order to define a clear goal, criteria were defined in order to evaluate
the finished product.
A primary focus were the before mentioned security considerations,
where the criteria need to be carefully chosen in order to provide
some guarantees in terms of safety. The second factor is to prove
that the provided functionality provides a benefit of some form.
This is achieved by implementing two real-world processes by Init7
utilizing the tool. Because the goal is to stay as generic as possible,
both in terms of vendor support and usefulness for businesses and individuals
other than Init7, these use cases were presented by the engineering team
after the implementation was complete. This is to ensure, that the implementation
is not biased for Init7 use cases and to show that there is a use for
the general networking community.

\subsection{\label{method:eval-sec}Security Criteria}

In order to argue security of a software system, threat actors need to be
defined which the system should be protected against. The threat actors
were identified with the help of the Init7 network engineering and operations
teams. Since they are responsible for the security of their systems and processes,
they are able to provide this information through their education and experience.
The following threat actors were identified.

\paragraph{External actors} These are defined as individuals or organizations
which are either not related to Init7 or are customers of Init7. As such,
these actors must have no access to either NetBox, any credentials or
the network infrastructure.

\paragraph{Unauthorized employees} Only a subset of network engineers need
access specific infrastructure and functionality. The system must be
able to differentiate between these sets of employees and enforce access
restrictions.

\paragraph{Authorized employees} While this group has the most access in
the system, considerations need to be made for potential mistakes.
This is usually implemented through validation and safeguards to eliminate
or reduce the potential impact for mistakes.

Using these threat actors, a list of security criteria was created,
listed and evaluated in \ref{eval-security}.

\subsection{Use case evaluation}

After the implementation, the network engineering team of Init7 presented
two scenarios which were to be implemented with the produced solution.
The network engineering team was then instructed on the general
functionality of the plugin and the use case was implemented in
a collaborative manner.

The collaborative process is used to gain an understanding of the perspective
of the network engineers in order to identify any potential issues
ranging from software to usability problems.