\chapter{\label{introduction}Introduction}
\thispagestyle{fancy}


\section{\label{introduction-current}State of current Technology}
% Describe current state of technology and solutions

Networking hardware makes out a huge part in our internet infrastructure. 
There is an increasing amount of hardware vendors varying in range of functionality. 
\glspl{ISP} are obvious customers of such hardware, but businesses of all sizes may have needs for all kinds of devices. 
This starts to pose the question, of how these devices shall be managed, especially when the need for scaling up arises. 
Many vendors supply configuration management solutions, also called \glspl{NMS} along their hardware offering, 
such as Cisco's ``Cisco DNA Center\cite{noauthor_cisco_nodate}'' or Juniper's ``Junos Space\cite{noauthor_junos_nodate}'' 
which might be the perfect fit in an environment wherein all hardware is obtained from the same vendor.
On the other hand, in mixed vendor environments, 
problems arise with these solutions as they primarily support their own hardware.

Since there's no ``one size fits all'' solution, one might want to come up with a custom solution.
In order to create a custom solution, one must know how devices are configured by \acrshort{NMS}s 
(or humans if that is automatable). There are numerous so-called network configuration protocols, 
such as \acrshort{SNMP}\cite{fedor_simple_1990}, CLI (e.g. via SSH or serial), NETCONF\cite{enns_network_2011} and others,
which can be used to remotely configure devices.

In order to configure devices, one must also know which devices exist within the infrastructure.
So called \acrfull{IRM} software is used to track the devices along with
information related to how they are connected and configured. 
While the above-mentioned \acrshort{NMS}'s usually provide such functionality, there are also
standalone solutions such as NetBox, which tracks everything from hardware to IP addresses.
Using the \acrshort{IRM} as the source of information, one would then be able to use a configuration protocol
to configure the devices according to one's needs.

\section{\label{introduction-goal}Goal / Aim}

% Describes the goal/aim of the work

The overarching goal is to provide a solution similar to an \acrshort{NMS} while utilizing open source
tools and software.
This will be achieved by extending NetBox with a plugin, which will allow an engineer to author configuration
templates which source their data from NetBox itself. 
These templates can then be applied to devices, wherein the plugin will generate the configuration and subsequently
apply it via a chosen configuration protocol.

As it is usually necessary to authenticate oneself in order to apply configuration to a device,
the issue of credential management and storage has to be solved.

\cn{Target open source space}
\cn{be vendor-agnostic}
\cn{compare transport protocols}
\cn{analyze compatibility}

% Separately describes the actual task to be performed, referencing the "Aufgabenstellung" of the BA



% Overview of the work

\section{\label{introduction-overview}Overview}



% Required knowledge for this paper

