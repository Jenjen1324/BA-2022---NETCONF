\chapter{\label{introduction}Introduction}
\thispagestyle{fancy}


\section{\label{introduction-current}State of current Technology}
% Describe current state of technology and solutions

Networking hardware makes out a huge part in our internet infrastructure. 
There are an increasing amount of hardware vendors varying in range of functionality. 
\glspl{ISP} are obvious customers of such hardware, but businesses of all sizes may have needs for all kinds of devices. 
This starts to pose the question, of how these devices shall be managed, especially when the need for scaling up arises. 
Many vendors supply configuration management solutions along their hardware offering, 
such as Cisco's ,,Cisco DNA Center``\cite{noauthor_cisco_nodate} or Juniper's ,,Junos Space``\cite{noauthor_junos_nodate} 
which might be the perfect fit in an environment wherein all hardware is obtained from the same vendor.

On the other hand, in mixed vendor environments, problems arise when a solution may only have limited support for other vendors.


\section{\label{introduction-goal}Goal / Aim}

% Describes the goal/aim of the work

%% Notes
 - Compare and choose configuration transports/protocols
   - Compatibility Matrix
   - Ease of use
   - Configuration portability between vendors


% Separately describes the actual task to be performed, referencing the "Aufgabenstellung" of the BA



% Overview of the work

% Required knowledge for this paper
