\clearpage
\null
\vfil % or it might be \null
\begin{center}\textbf{Abstract}\end{center}

Managing the configuration of network devices is a complex topic and relevant
to both small and large businesses. After identifying a lot of manual processes within Init7 AG,
an effort to find better solutions was started. There are a lot of different protocols and
software solutions attempting to solve this problem, but most of them are either
limited in functionality or are solutions by hardware vendors which only support
their own networking gear.
As Init7 is an open-source minded company, uses multiple vendors for their hardware
and needs such a solution, the idea started of creating a custom implementation
of a network management system. The goal was to utilize open source tooling
and create a system that focuses on being flexible and vendor-agnostic.
Init7 already uses NetBox, an open source DCIM/IPAM, for inventory and documentation purposes.
This sparked the idea to build a configuration management solution on top of NetBox.

Following the initial idea, a collaboration between ZHAW and Init7 was formed to
create a proof of concept within the scope of this thesis. Together with Init7
the goals were outlined, and a software project plan was created.

NETCONF/YANG was chosen as the method of configuration, as it provides a clean
interface for integration and is also supported by many major vendors.
Focus was also laid onto the security of the solution, as it needs to access
critical network infrastructure.

The resulting proof of concept, proved to be useful in real-world applications
within Init7 business use-cases. Two complete services were implemented and were
verified by the Init7 network engineering team. Since the project was a success,
there is a high chance that Init7 will continue work on this project to make it
ready for a production environment.

\clearpage
\null
\vfil % or it might be \null
\begin{center}\textbf{Zusammenfassung}\end{center}

Die Verwaltung der Konfiguration von Netzwerkgeräten ist ein komplexes 
Thema und relevant sowohl für kleine als auch für grosse Unternehmen. 
Nach der Identifizierung vieler manueller Prozesse innerhalb der Init7 AG, 
wurde ein Versuch gestartet, bessere Lösungen zu finden. Es gibt eine Vielzahl 
von Protokollen und Softwarelösungen, die versuchen, dieses Problem zu lösen, 
aber die meisten davon sind entweder aber die meisten von ihnen sind entweder 
in ihrer Funktionalität eingeschränkt oder sind Lösungen von Hardwareherstellern, 
die nur ihre eigenen Netzwerkgeräte unterstützen. ihre eigene Netzwerkausrüstung unterstützen.
Da Init7 ein Open-Source-Unternehmen ist, das seine Hardware von mehreren Anbietern 
bezieht und eine solche Lösung benötigt, entstand die Idee, eine eigene Implementierung 
eines Netzwerkmanagementsystems zu entwickeln. Das Ziel war es, Open-Source-Tools zu verwenden 
und ein System zu schaffen, das flexibel und herstellerunabhängig ist. Init7 verwendet bereits 
NetBox, ein Open Source DCIM/IPAM, für Inventarisierungs- und Dokumentationszwecke. 
Daraus entstand die Idee, eine Konfigurationsmanagementlösung auf der Grundlage von NetBox zu entwickeln.

Nach der ersten Idee kam es zu einer Zusammenarbeit zwischen der ZHAW und Init7, 
um im Rahmen dieser Arbeit einen Proof of Concept zu erstellen. 
Gemeinsam mit Init7 wurden die Ziele umrissen und ein Software-Projektplan erstellt.

Als Konfigurationsmethode wurde NETCONF/YANG gewählt, da es eine saubere Schnittstelle 
für die Integration bietet und zudem von vielen grossen Anbietern unterstützt wird. 
Ein weiterer Schwerpunkt lag auf der Sicherheit der Lösung, da sie auf kritische Netzwerkinfrastruktur
zugreifen muss. kritische Netzwerkinfrastruktur zugreifen muss.

Der daraus resultierende Konzeptnachweis erwies sich in realen Anwendungen als nützlich im Rahmen 
von Init7-Geschäftsanwendungen. Zwei vollständige Dienste wurden implementiert und wurden vom 
Init7-Netzwerktechnikteam verifiziert. Da das Projekt ein Erfolg war, besteht eine hohe Wahrscheinlichkeit, 
dass Init7 die Arbeit an diesem Projekt fortsetzen wird, um es für eine Produktionsumgebung bereit zu machen.
