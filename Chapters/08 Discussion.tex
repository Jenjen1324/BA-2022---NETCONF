\chapter{\label{discussion}Discussion}
\thispagestyle{fancy}

\section{Achieved goals}

To summarize, the goals of providing a flexible and vendor-agnostic configuration was reached
partially. For a proof of concept work being able to model
real world offerings which are high in complexity, and subsequently configure them correctly,
proves that combining NetBox with NETCONF can create a powerful \acrshort{OSS}.

While it was verified that quite a few other vendors support NETCONF as well, no concrete
testing could be performed due to time and also equipment constraints.

After working with together with the Init7 engineering staff and later also presenting
the finished product, the consensus was to pursue this avenue further. As the project
is open source, this may also bring future benefits to the general networking community.

Some flaws were found in the security of the work, which have to be addressed before
it is used in a production environment. While one could describe the finding as an
oversight, a security audit may be necessary in order to ensure that no other
oversights were missed, especially for such a sensitive system.

\section{Problems during development}

Major time blockers were inconsistencies especially with Cisco's IOS-XE software.
Several bugs in their NETCONF implementation were identified, such as sections of configuration
entirely missing from the YANG models, \icode{<get-config>} operations not returning the full data,
and even \icode{<edit-config>} nodes which are documented in the YANG model but proved to be non-functional
in practice. Several support tickets were opened with Cisco, without response yet.

