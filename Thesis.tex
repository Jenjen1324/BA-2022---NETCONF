%%****************************************************************************
%** Copyright 2002, 2003 by Lukas Ruf, <ruf@topsy.net>
%** Information is provided under the terms of the
%** GNU Free Documentation License <http://www.gnu.org/copyleft/fdl.html>
%** Fairness: Cite the source of information, visit <http://www.topsy.net>
%****************************************************************************
%** Last Modification: 2005-07-11 1600
%** 2005-07-11	Bernhard Tellenbach
%**							Switched default document class to: book
%**							Added \include{appendix.tex}
%****************************************************************************

%\documentclass[10pt,final,a4paper,twoside]{book}
%\documentclass[10pt,draft,a4paper,oneside]{report}
\documentclass[headings=optiontohead,10pt,final,a4paper,oneside]{scrreprt}
%\documentclass[10pt,draft,a4paper,oneside]{article}

%**Latex Master Document*********

%** preamble.tex: here all the document-wide settings
%                 are defined
\RequirePackage{times}

\usepackage[english]{babel}
%-% \usepackage[german]{babel}
%\usepackage[ngerman]{babel}

\usepackage[utf8]{inputenc}
\usepackage[T1]{fontenc}
\usepackage{textcomp}
\usepackage{type1cm}
\usepackage[table,xcdraw]{xcolor}
\usepackage{a4}

\usepackage{graphicx}
\graphicspath{{Figures/},{Pictures/}}
\usepackage{subfigure}

\usepackage{fancyhdr}
\usepackage{fancybox}
\usepackage{hyperref}

\usepackage{float}
\usepackage{longtable}
\usepackage{paralist}
\usepackage{url}
\usepackage{lscape}
\usepackage{moreverb}

\usepackage{makecell}
\usepackage{multirow}

\usepackage{tabularx}
\usepackage{xltabular}

\usepackage{nomencl}
  \let\abbrev\nomenclature
  \renewcommand{\nomname}{List of Abbrevations}
  \setlength{\nomlabelwidth}{.25\hsize}
  \renewcommand{\nomlabel}[1]{#1 \dotfill}
  \setlength{\nomitemsep}{-\parsep}
  %For old nomencl package, uncomment this:
  \makeglossary 
  %For new nomencl package, uncomment this:
  %\makenomenclature

\usepackage[normalem]{ulem}
  \newcommand{\markup}[1]{\uline{#1}}
  
   
\usepackage{ae,aecompl}


\usepackage[first,bottomafter,light]{draftcopy}
\draftcopyName{Draft v0.1}{120}

\addtolength{\textwidth}{2cm}
\addtolength{\textheight}{2cm}
\addtolength{\oddsidemargin}{-1.0cm}
\addtolength{\evensidemargin}{-1.0cm}
\addtolength{\topmargin}{-1.5cm}

%% No Serifs: Put comment markers in front of the next three lines otherwise
\renewcommand{\ttdefault}{cmtt}
\renewcommand{\rmdefault}{phv}  % Helvetica for roman type as well as sf
\renewcommand{\ttdefault}{pcr}  % use Courier for fixed pitch, if needed

\newcommand{\?}{\discretionary{/}{}{/}}
\newcommand{\liter}[0]{/home/ruf/Lib/Bibl/}
\newcommand{\fref}[1]{\mbox{Figure~\ref{#1}}}

\pagestyle{fancy}
%%-lpr Note: 'chapters' are defined for 'book's only
%%-lpr       in articles, we make use of sections only
\renewcommand{\chaptermark}[1]{\markboth{#1}{}}
\renewcommand{\sectionmark}[1]{\markright{\thesection\ #1}}
\fancyhf{}

%\fancyhead[LE,RO]{\bfseries\thepage}
%\fancyhead[LO]{\bfseries\rightmark}
%\fancyhead[RE]{\bfseries\leftmark}
\fancyhead[R]{\leftmark}
\fancyhead[L]{\rightmark}
\fancyfoot[R]{\thepage}
\fancyfoot[C]{ZHAW SoE}
\fancyfoot[L]{Bachelor Thesis (IT)}

\renewcommand{\headrulewidth}{0.5pt}
\renewcommand{\footrulewidth}{0.5pt}
\addtolength{\headheight}{0.5pt}
\fancypagestyle{plain}{%
   \fancyhf{}
   \fancyfoot[C]{\bfseries \thepage}
   \fancyhead{}%get rid of headers on plain pages
   \renewcommand{\headrulewidth}{0pt} % an the line
}
\newcommand{\clearemptydoublepage}{\newpage{\pagestyle{empty}\cleardoublepage}}

\setlength{\parindent}{0in}



\hyphenation{Strea-men}
\hyphenation{Apps}
\hyphenation{Threat}
\hyphenation{Init}

\newcommand{\Appendix}[2][?]
{
  \refstepcounter{section}
  \addcontentsline{toc}{appendix}
  {
    \protect\numberline{\appendixname~\thesection} %1
  }
  {
    \flushright\large\bfseries\appendixname\ \thesection\par
    \nohypens\centering#1\par
  }
  \vspace{\baselineskip}
}

\let\margin\marginpar
\newcommand\myMargin[1]{\margin{\raggedright\scriptsize #1}}
\renewcommand{\marginpar}[1]{\myMargin{#1}}

\newcommand\CHECK{\myMargin{CHECK}}
\newcommand\NEW{\myMargin{NEW}}

\usepackage[acronym,numberedsection=autolabel,section=section]{glossaries}

\makeglossaries

% \newacronym{<id>}{<short>}{<long>}

%\newglossaryentry{glos:mpegts}{
% name=MPEG Transport Stream,
% description={Von der Moving Picture Experts Group entwickeltes Transportstream-Protokoll für Video/Audio-Übertragung}
%}


%\useshorthands*{"}
%\addto\extrasenglish{\languageshorthands{ngerman}}

\usepackage{csquotes}
\usepackage[
backend=biber,
bibstyle=ieee,
style=numeric,
sorting=none
]{biblatex}
\addbibresource{references.bib}


\usepackage{pdfpages}
\usepackage{rotating}
\usepackage{pdflscape}

%% Adds another nesting level of numbering for sections
\setcounter{secnumdepth}{3}

\usepackage{helvet}
%\renewcommand*{\familydefault}{\sfdefault}

\usepackage{color,soul}

\usepackage{scrhack}

\usepackage[section,numbib,nottoc]{tocbibind}
\renewcommand{\listoffigures}{
    \tocsection
    \tocfile{\listfigurename}{lof}
}
\renewcommand{\listoftables}{
    \tocsection
    \tocfile{\listtablename}{lot}
}

%********************************

%** begin the document environment
\begin{document}

\frenchspacing
\sloppy

%** Titel.tex: Title page to be printed first

\includepdf[pages=-]{../Attachments/SoE-INIT_Bachelor-Titelblatt-Vorlage_en.pdf}
\includepdf[pages=-]{../Attachments/Erklaerung_BA_digital-_EN.pdf}


%** environments.tex: Predefined Environments
\include{environments}
%**Documentation****************

%** Zusammenfassung.tex
\clearpage
\null
\vfil % or it might be \null
\begin{center}\textbf{Abstract}\end{center}

Managing the configuration of network devices is a complex topic and relevant
to both small and large businesses. After identifying a lot of manual processes within Init7 AG,
an effort to find better solutions was started. There are a lot of different protocols and
software solutions attempting to solve this problem, but most of them are either
limited in functionality or are solutions by hardware vendors which only support
their own networking gear.
As Init7 is an open-source minded company, uses multiple vendors for their hardware
and needs such a solution, the idea started of creating a custom implementation
of a network management system. The goal was to utilize open source tooling
and create a system that focuses on being flexible and vendor-agnostic.
Init7 already uses NetBox, an open source DCIM/IPAM, for inventory and documentation purposes.
This sparked the idea to build a configuration management solution on top of NetBox.

Following the initial idea, a collaboration between ZHAW and Init7 was formed to
create a proof of concept within the scope of this thesis. Together with Init7
the goals were outlined, and a software project plan was created.

NETCONF/YANG was chosen as the method of configuration, as it provides a clean
interface for integration and is also supported by many major vendors.
Focus was also laid onto the security of the solution, as it needs to access
critical network infrastructure.

The resulting proof of concept, proved to be useful in real-world applications
within Init7 business use-cases. Two complete services were implemented and were
verified by the Init7 network engineering team. Since the project was a success,
there is a high chance that Init7 will continue work on this project to make it
ready for a production environment.

\clearpage
\null
\vfil % or it might be \null
\begin{center}\textbf{Zusammenfassung}\end{center}

Die Verwaltung der Konfiguration von Netzwerkgeräten ist ein komplexes 
Thema und relevant sowohl für kleine als auch für grosse Unternehmen. 
Nach der Identifizierung vieler manueller Prozesse innerhalb der Init7 AG, 
wurde ein Versuch gestartet, bessere Lösungen zu finden. Es gibt eine Vielzahl 
von Protokollen und Softwarelösungen, die versuchen, dieses Problem zu lösen, 
aber die meisten davon sind entweder aber die meisten von ihnen sind entweder 
in ihrer Funktionalität eingeschränkt oder sind Lösungen von Hardwareherstellern, 
die nur ihre eigenen Netzwerkgeräte unterstützen. ihre eigene Netzwerkausrüstung unterstützen.
Da Init7 ein Open-Source-Unternehmen ist, das seine Hardware von mehreren Anbietern 
bezieht und eine solche Lösung benötigt, entstand die Idee, eine eigene Implementierung 
eines Netzwerkmanagementsystems zu entwickeln. Das Ziel war es, Open-Source-Tools zu verwenden 
und ein System zu schaffen, das flexibel und herstellerunabhängig ist. Init7 verwendet bereits 
NetBox, ein Open Source DCIM/IPAM, für Inventarisierungs- und Dokumentationszwecke. 
Daraus entstand die Idee, eine Konfigurationsmanagementlösung auf der Grundlage von NetBox zu entwickeln.

Nach der ersten Idee kam es zu einer Zusammenarbeit zwischen der ZHAW und Init7, 
um im Rahmen dieser Arbeit einen Proof of Concept zu erstellen. 
Gemeinsam mit Init7 wurden die Ziele umrissen und ein Software-Projektplan erstellt.

Als Konfigurationsmethode wurde NETCONF/YANG gewählt, da es eine saubere Schnittstelle 
für die Integration bietet und zudem von vielen grossen Anbietern unterstützt wird. 
Ein weiterer Schwerpunkt lag auf der Sicherheit der Lösung, da sie auf kritische Netzwerkinfrastruktur
zugreifen muss. kritische Netzwerkinfrastruktur zugreifen muss.

Der daraus resultierende Konzeptnachweis erwies sich in realen Anwendungen als nützlich im Rahmen 
von Init7-Geschäftsanwendungen. Zwei vollständige Dienste wurden implementiert und wurden vom 
Init7-Netzwerktechnikteam verifiziert. Da das Projekt ein Erfolg war, besteht eine hohe Wahrscheinlichkeit, 
dass Init7 die Arbeit an diesem Projekt fortsetzen wird, um es für eine Produktionsumgebung bereit zu machen.


%** Vorwort.tex
\thispagestyle{empty}
\clearpage
\null
\vfil % or it might be \null
\begin{center}\textbf{Foreword}\end{center}

This bachelor thesis was written as part of our part-time studies in computer 
science at the ZHAW School of Engineering in Winterthur. 
The idea for this thesis came from Jens Vogler, who works part-time at Init7 (Switzerland) AG 
as the software engineering team lead. 

The topic of this thesis was already in discussion within Init7 and the opportunity
to write a thesis presented itself.
The work was supervised by Dr. Gürkan Gür. He works at the ZHAW in the research area Information Security
and as a lecturer for IT security. 
We would like to thank him for his support during our work and his flexibility, 
especially during stressful times as the sudden change to a leadership position within Init7
presented challenges which were not anticipated. 
We would also like to thank Sam Aschwanden, Lino De Moragon, Thomas Fritz, Pascal Gloor and the
rest of the Init7 network engineering team. 
They all provided valuable inputs and help throughout the thesis.

%** Table of Contents
\tableofcontents

%** Einleitung.tex: 
% - Nennt bestehende Arbeiten/Literatur zum Thema
% - Stand der Technik: Bisherige Lösungen des Problems und deren Grenzen
% - (Nennt kurz den Industriepartner und/oder weitere Kooperationspartner und dessen/deren Interesse am Thema Fragestellung)
% - Formuliert das Ziel der Arbeit
% - Verweist auf die offizielle Aufgabenstellung des/der Dozierenden im Anhang
% - (Pflichtenheft, Spezifikation)
% - Übersicht über die Arbeit: stellt die folgenden Teile der Arbeit kurz vor
% - (Angaben zum Zielpublikum: nennt das für die Arbeit vorausgesetzte Wissen)
% - (Terminologie: Definiert die in der Arbeit verwendeten Begriffe)
\chapter{\label{introduction}Introduction}
\thispagestyle{fancy}


\section{\label{introduction-current}State of current Technology}
% Describe current state of technology and solutions

Networking hardware makes out a huge part in our internet infrastructure. 
There is an increasing amount of hardware vendors varying in range of functionality. 
\glspl{ISP} are obvious customers of such hardware, but businesses of all sizes may have needs for all kinds of devices. 
This starts to pose the question, of how these devices shall be managed, especially when the need for scaling up arises. 
Many vendors supply configuration management solutions, also called \glspl{NMS} along their hardware offering, 
such as Cisco's ``Cisco DNA Center\cite{noauthor_cisco_nodate}'' or Juniper's ``Junos Space\cite{noauthor_junos_nodate}'' 
which might be the perfect fit in an environment wherein all hardware is obtained from the same vendor.
On the other hand, in mixed vendor environments, 
problems arise with these solutions as they primarily support their own hardware.

Since there's no ``one size fits all'' solution, one might want to come up with a custom solution.
In order to create a custom solution, one must know how devices are configured by \acrshort{NMS}s 
(or humans if that is automatable). There are numerous so-called network configuration protocols, 
such as \acrshort{SNMP}\cite{fedor_simple_1990}, CLI (e.g. via SSH or serial), NETCONF\cite{enns_network_2011} and others,
which can be used to remotely configure devices.

In order to configure devices, one must also know which devices exist within the infrastructure.
So called \acrfull{IRM} software is used to track the devices along with
information related to how they are connected and configured. 
While the above-mentioned \acrshort{NMS}'s usually provide such functionality, there are also
standalone solutions such as NetBox, which tracks everything from hardware to IP addresses.
Using the \acrshort{IRM} as the source of information, one would then be able to use a configuration protocol
to configure the devices according to one's needs.

\section{\label{introduction-goal}Goal / Aim}

% Describes the goal/aim of the work

The overarching goal is to provide a solution similar to an \acrshort{NMS} while utilizing open source
tools and software.
This will be achieved by extending NetBox with a plugin, which will allow an engineer to author configuration
templates which source their data from NetBox itself. 
These templates can then be applied to devices, wherein the plugin will generate the configuration and subsequently
apply it via a chosen configuration protocol.

As it is usually necessary to authenticate oneself in order to apply configuration to a device,
the issue of credential management and storage has to be solved.

One primary objective is to attempt to be as vendor-agnostic as possible. This is hoped to be achieved
through using standardized protocols and conventions.

\subsection{Tasks}

\begin{itemize}
    \item Plan the project in a structured manner utilizing software engineering principles
    \item Evaluate different configuration protocols and decide on one to use
    \item Consider security through threat models and security evaluations
    \item Create a NetBox plugin which enables an engineer to implement products and services
    in a multivendor infrastructure
\end{itemize}



%** TheoretischeGrundlagen.tex:
\chapter{\label{theory}Theory}
\thispagestyle{fancy}


%** Vorgehen.tex:
% - Beschreibt die Grundüberlegungen der realisierten Lösung (Konstruktion/Entwurf) und die Realisierung als Simulation, als Prototyp oder als Software-Komponente etc.
% - (Definiert Messgrössen, beschreibt Mess- oder Versuchsaufbau, beschreibt und dokumentiert Durchführung der Messungen/Versuche)
% - (Experimente)
% - (Lösungsweg)
% - (Modell)
% - (Eingesetzte Software)
% - (Tests und Validierung)
\chapter{\label{methods}Methods}
\thispagestyle{fancy}

\section{Literature Research}

- Lots of papers on protocols
- Difficulty in finding concrete cases
- Lots of stuff is propiritary
  - Open Source solutions are very fragemented
  - Makes it hard to find examples for working solutions

\section{Software Engineering Method}

\section{Evalution Method}

%** Resultate.tex:
% - Zusammenfassung der Resultate
\chapter{\label{results}Results}
\thispagestyle{fancy}

\section{Current State}

\subsection{Init7 (Schweiz) AG}

\subsection{Rack Black Association}

\section{Requirements Analysis}

\section{Application structure / Architecture}

To bridge the gap between the inventory management provided by NetBox and communicating with the actual hardware,
the solution is divided into layers.
In order to facilitate authentication and authorization there is the ,,Credential Access Manager'' which communicates
with an external credential store. The access manager provides a user interface to enter and manage credentials, as well
as restricting who has access to these credentials by integrating into the NetBox permission system.
The ,,Device Context Provider'' enables a user to define what information for a specific device shall be pulled.
Using the existing GraphQL API provided by NetBox, the user can easily identify what information is available and how it is
accessed. After the context data is defined it will subsequently be used in the ,,Configuration Template Editor''
where the user writes a template using the previously defined context data. This template can then be applied to
a class of devices utilizing constraints such as tags or locations.
Lastly, the ,,Configuration Transport'' module pulls the configuration template and the credentials for a specific device
and pushes it to the actual device.

\subsection{Credential Access Manager}

\subsection{Device Context Provider}

\subsection{Configuration Template Editor}

\subsection{Configuration Transport}

%** DiskussionAusblick.tex:
% - Bespricht die erzielten Ergebnisse bezüglich ihrer Erwartbarkeit, Aussagekraft und Relevanz
% - Interpretation und Validierung der Resultate
% - Rückblick auf Aufgabenstellung, erreicht bzw. nicht erreicht
% - Legt dar, wie an die Resultate (konkret vom Industriepartner oder weiteren Forschungsarbeiten; allgemein) angeschlossen werden kann; legt dar, welche Chancen die Resultate bieten.
\chapter{\label{discussion}Discussion}
\thispagestyle{fancy}

\section{Achieved goals}

To summarize, the goals of providing a flexible and vendor-agnostic configuration was reached
partially. For a proof of concept work being able to model
real world offerings which are high in complexity, and subsequently configure them correctly,
proves that combining NetBox with NETCONF can create a powerful \acrshort{OSS}.

While it was verified that quite a few other vendors support NETCONF as well, no concrete
testing could be performed due to time and also equipment constraints.

After working with together with the Init7 engineering staff and later also presenting
the finished product, the consensus was to pursue this avenue further. As the project
is open source, this may also bring future benefits to the general networking community.

Some flaws were found in the security of the work, which have to be addressed before
it is used in a production environment. While one could describe the finding as an
oversight, a security audit may be necessary in order to ensure that no other
oversights were missed, especially for such a sensitive system.

\section{Problems during development}

Major time blockers were inconsistencies especially with Cisco's IOS-XE software.
Several bugs in their NETCONF implementation were identified, such as sections of configuration
entirely missing from the YANG models, \icode{<get-config>} operations not returning the full data,
and even \icode{<edit-config>} nodes which are documented in the YANG model but proved to be non-functional
in practice. Several support tickets were opened with Cisco, without response yet.



%** Verzeichnisse.tex:
\chapter{\label{references}References}
\thispagestyle{fancy}

\raggedright
\printbibliography[title={Bibiliogrpahy},heading=subbibnumbered]
% [title={Literaturverzeichnis},heading=subbibnumbered] würde nummerieren

\pagebreak
\printglossary

%** Table of Figures
\pagebreak
\listoffigures

\pagebreak
\printglossary[type=\acronymtype]

%** Anhang.tex:

\chapter{\label{appendix}Appendix}
\thispagestyle{fancy}

%** end the document environment
\end{document}
